% ********** Chapter 1 **********
\chapter{Introductie}
\label{sec:Hoofdstuk 1}

Een deel van de cursus Gevorderde Algoritmen en Datastructuren is een praticum, deze bestaat uit twee opdrachten. De eerste opdracht staat in dit document beschreven. Het doel van deze opdracht luidt: \textit{"ervaring opdoen met het analyseren en vergelijken van zoekalgoritmen. De opdracht sluit aan bij Chapter 3 van het leerboek Algorithm Design van Goodrich en Tamassia.
Het is de bedoeling om twee programma�s te ontwikkelen, waarmee je kunt bepalen welke
datastructuur beter is om een doorzoekbaar geordend lexicon (searchable ordered dictionary) te implementeren:
een conventionele binaire zoekboom, of een van de volgende alternatieve datastructuren:
AVL-boom, (2,4)-boom, splay tree, skip list. De programma�s dienen een tekstbestand woord voor
woord te lezen, waarbij elk nieuw woord wordt toegevoegd aan de datastructuur. Elk woord komt dus
slechts �e�enmaal voor in de datastructuur."}\\
\\
Een aantal van deze datastructuren staat beschreven in dit document, naast de beschrijving van de datastructuur is ook de uitkomst van de analyses die zijn uitgevoerd beschreven. In deze analyses is gekeken naar het aantal stringvergelijkingen, het aantal toekenningen in de code en de tijd die nodig is bij bepaalde handelingen.\\
\\
Als eerste zal in hoofdstuk 2 de conventionele binaire zoekboom behandeld worden. Hierna wordt gekeken naar gebalanceerde bomen, namelijk de AVL-boom (hoofdstuk 4) en de (2,4)-boom (hoofdstuk 5). Als laatste wordt in hoofdstuk 6 een alternatieve datastructuur behandeld, de skip list.\\
\\
\section{Verwachting}.........................
....................
....................

% ********** End of chapter **********
